\documentclass[
    11pt,
    a4paper
]{extarticle}

\usepackage[utf8]{inputenc}
\usepackage{listings}

\usepackage[
    a4paper,
    left = 10mm,
    right = 10mm,
    top = 5mm,
    bottom = 5mm,
    bindingoffset = 0cm,
    columnsep = 1cm
]{geometry}

\usepackage{amsmath,amssymb,amsthm}
\usepackage[cp1251]{inputenc}
\usepackage[T2A]{fontenc}
\usepackage[russian]{babel}
%\usepackage{nopageno,comment}
\usepackage{indentfirst}
\usepackage{cmap}
\usepackage{ifthen}
\usepackage{tikz}
\usepackage{wrapfig}

\newcommand{\Sum}{\displaystyle\sum\limits}
\newcommand{\Max}{\max\limits}
\newcommand{\Min}{\min\limits}
\newcommand{\fromto}[3]{{#1}=\overline{{#2},\,{#3}}}
\newcommand{\floor}[1]{\left\lfloor{#1}\right\rfloor}
\newcommand{\ceil}[1]{\left\lceil{#1}\right\rceil}
\newcommand{\NN}{\mathbb{N}}
\newcommand{\RR}{\mathbb{R}}
\newcommand{\tild}{\widetilde}
\renewcommand{\le}{\leqslant}
\renewcommand{\ge}{\geqslant}
\renewcommand{\hat}{\widehat}
\renewcommand{\emptyset}{\varnothing}
\renewcommand{\epsilon}{\varepsilon}
\newcommand{\ol}{\overline}


\newcounter{task}

\newcommand{\printscore}[1]{%
\ifthenelse{\equal{#1}{1}}{(1 балл).}{}%
\ifthenelse{\equal{#1}{2}}{(2 балла).}{}%
\ifthenelse{\equal{#1}{3}}{(3 балла).}{}%
\ifthenelse{\equal{#1}{4}}{(4 балла).}{}%
\ifthenelse{\equal{#1}{5}}{(5 баллов).}{}%
\ifthenelse{\equal{#1}{6}}{(6 баллов).}{}%
}

\newcommand{\task}[2]{\par\noindent\stepcounter{task}{\bf Задача~\arabic{task}.~\printscore{#1}} {#2}\vskip 6pt}

\newcommand*{\hm}[1]{#1\nobreak\discretionary{}{\hbox{$#1$}}{}}

%\nofiles
\begin{document}

\centerline{\large \bf Первое домашнее задание по курсу}
\centerline{\large \bf <<Дискретный анализ и теория вероятностей>>}\bigskip

\textbf{Внимание!} В задачах этого домашнего задания решения, состоящие только из одного ответа, не принимаются и ставится $0$ баллов за задачу, даже если этот ответ верный. Каждое решение должно быть пояснено! В условиях задач все люди (если они там, конечно, встречаются) считаются разными, ибо каждый человек - личность! Также обязательно учтите правила оформления работ, описанные на странице курса на ШАД-вики!

\task{2}{Сколькими способами можно переставить цифры числа $12345321$ таким образом, чтобы цифры $2,$ $2,$ $4$ и $5$ обязательно стояли между двумя цифрами $1$?}

\textbf{Решение}

Количество способов расставить цифры $2,$ $2,$ $4,$ $5$ равно $P(2, 1, 1) = \frac{4!}{(2! \cdot 1! \cdot 1!)}$. Пусть X - перестановки из цифр $2$, $2$, $4$, $5$. Тогда возможны $6$ случаев расстановки оставшихся цифр $3$:
$33$X, X$33$, $3$X$3$, $3(X\cup3)$, $(X\cup3)3$, $X\cup3\cup3$. Первые три, очевидно, дают $P(2, 1, 1)$. Следующие две дают $P(2, 1, 1, 1)$ каждая, и, наконец, последняя - $P(2, 2, 1, 1)$.\\
Итого: $P(2, 1, 1) \cdot 3 + 2 \cdot P(2, 1, 1, 1) + P(2, 2, 1, 1) = 336$.


\task{2}{Сколькими способами можно переставить цифры числа $12345678$ так, чтобы перестановка удовлетворяла перечисленным ниже четырём ограничениям?
\begin{itemize}
  \item[1.] Цифры $2,6$ и $8$ стоят в любом порядке и занимают соседние позиции в получившемся числе;
  \item[2.] Цифры $1,3$ и $7$ стоят в любом порядке и между этими трёмя цифрами стоит не более одной другой цифры;
  \item[3.] Цифры $1$ и $8$ занимают позиции разной чётности в получившемся числе. Например, если цифра $1$ стоит на $2$-й позиции после перестановки, то цифру $8$ можно поставить при перестановке на одну из следующих позиций: $1,3,5,7$;
  \item[4.] Цифры $5$ и $6$ не стоят рядом в получившимся числе.
\end{itemize}}

\textbf{Решение (код)}

\begin{lstlisting}
from itertools import permutations

elms = [1, 2, 3, 4, 5, 6, 7, 8]

def check_268(elms):
    i_2, i_6, i_8 = elms.index(2), elms.index(6), elms.index(8)
    return abs(i_2 - i_6) <= 2 and abs(i_6-i_8) <= 2 and abs(i_8 - i_2) <= 2

def check_137(elms):
    idxs = sorted([elms.index(1), elms.index(3), elms.index(7)])
    return idxs[2] - idxs[0] <= 3

def check_18(elms):
    return elms.index(1) % 2 != elms.index(8) % 2

def check_56(elms):
    return abs(elms.index(5) - elms.index(6)) > 1


def check(elms):
    return check_268(elms) and check_137(elms) and check_18(elms) and \
        check_56(elms)

print sum(map(lambda x: 1 if x else 0, map(check, permutations(elms))))
\end{lstlisting}

Что дает ответ в $760$


\task{2}{Сколько $4$-значных чисел можно составить, если в этом числе может быть не более $2$-x цифр $1$, не более $2$-x цифр $2$, не более $2$-x цифр $4$ и не более $4$-x цифр $3$, а других цифр в этом числе нет? }

\textbf{Решение}

Количество всех $4$-значных чисел из цифр $1234$ равно $4^4$. Количество чисел вида $xxxy$, где $3$ одинаковые из набора $1$, $2$, $4$, равно $C_3^1 \cdot C_2^1 \cdot P(3, 1)$. Количество чисел вида $xxxx$ из набора $1$, $2$, $4$ равно $C_3^1$. Остальные варианты подходят.\\
Итого $4^4 - C_3^1 \cdot C_2^1 \cdot P(3, 1) - C_3^1$ = $217$.


\task{2}{Найдите число слов, составленных из букв русского алфавита, длины $8$, не содержащих подслов ``куб'' и ``бак''. В алфавите русского языка содержится $33$ буквы. Подслово - это слово, составленное из части букв слова, идущих подряд. Слово --- последовательность букв алфавита без учёта его осмысленности. Пример слова \emph{содержащего} подслова ``куб'' и ``бак'': ``кубакааа''.

\textbf{Решение}

Всего $8$-буквенных слов $33^8$. Подстроки ``бак'' и ``куб'' можно разместить $33^5\cdot6$ способами из-за выбора начальной позиции (х$2$ в ответ, так как слов $2$) и количество выбора букв на оставшиеся (Множество $1$). Если есть $2$ слова и не накладываются (повторение одного слова не наложится, очевидно), например ``кубкуб'' или ``куббак'', то это даст $33^2\cdot6$ - опять же из-за начальной позиции, в ответ уйдет с х$4$, потому что $4$ варианта выбора этих двух слов, а $6$ - потому что оставшиеся $2$ буквы можно расставить в виде $1133$, $233$, $1313$, $323$, $3131$, $332$ ($1$, $2$ - это оставшиеся буквы, $3$ - исходные слова) (Множество $2$). Далее, если есть оба слова и они накладываются, как ``кубак'' и ``бакуб'', то это даст еще $33^3\cdot4$ каждый (Множество $3$). Когда пересечено $3$ слова и все пересекаются: ``кубакуб'' или ``бакубак'' (Множество $4$): $2*2*33=132$ - $2$ всего $2$ слова и $2$ - оставшаяся буква слева/справа. И, наконец, множество $5$ - случай, когда $3$ слова и $2$ из них пересекаются: ``кубакбак'', ``кубаккуб'', ``бакуббак'', ``бакубкуб'', ``баккубак'', ``кубкубак'', ``бакбакуб'', ``куббакуб'': просто $8$.\\
Итого по формуле включений исключений: $33^8-33^5\cdot6\cdot2+(33^2\cdot6\cdot4+33^3\cdot4\cdot2)-(33\cdot2\cdot2+8)$

\textbf{Внимание!} При оформлении задачи учтите пункт $5$ правил на странице курса на ШАД-вики. Если он проигнорирован, полный балл Вы не получите!}

\task{1}{Найдите коэффициент при $x^{29}$ для $(1 + x^5 + x^7 + x^9)^{1000}$.}

\textbf{Решение}

При перемножении различных степеней $х$ будет получаться итоговая степень, равная: $5a + 7b + 9c$. Переберем (начиная с $9$, $7$, $5$ степени вхождения и подберем остальные параметры):
$c = 3$: $\oslash$\\
$c = 2$: $\oslash$\\
$c = 1$, $a = 4$;\\
$c = 0$, $b = 2$, $a = 3$.\\
Итого, $C_{1000}^1 \cdot C_{999}^4 + C_{1000}^2 \cdot C_{998}^3$.

\task{}{На клетчатой бумаге изображен квадрат, каждая сторона которого умещает ровно $n$ клеток. Фигуры, отличающиеся размерами и/или положением в клетчатом прямоугольнике считаются различными. Ответ необходимо дать в виде суммы, содержащей фиксированное (не зависящее от $n$) количество слагаемых, за ответ другого вида полный балл ставиться не будет. Условимся также считать, что $n$ такое, что хотя бы одна фигура каждого, описанного ниже вида, в этот квадрат влезает.
\begin{itemize}
\item[а)]\textbf{(2 балла)} Сколько в этом квадрате можно нарисовать различных фигур $F_1$? (см. рис. $2$). Фигурой $F_1$ следует считать фигуру, состоящую из горизонтальной ``перекладины'' толщины в 1 клетку, к концам которой снизу крепятся две одинаковых перекладины высоты не менее $2$ клеток и толщины в $1$ клетку - расстояние между этим перекладинами не менее $2$ клекок, аналогично сверху к концам горизонтальной перекладины крепятся две одинаковые вертикальные перекладины высоты не менее $1$ клетки, повороты учитывать не следует. Концы горизонтальной перекладины обозначены зеленым цветом. (Приоритет отдавайте рисунку, а не описанию фигуры. Если описание противоречит рисунку - давайте знать, но правильная интерпретация задачи всегда нарисована на рисунке.)

\textbf{Решение}

Перекладину выберем $C_{n-2}^2$ способами, $n-2$ - потому что ее длина минимум $2$, и $2$ точки как ограничения слева/справа. Границы вертикальных линий выберем $C_{n-1}^3$ - $n-1$ - потому что расстояние снизу минимум $2$, а $3$ - $3$ горизонтали (перекладина и верхняя/нижняя точка). Собственно, вот и вся настройка.\\

В итоге $C_{n-2}^2+C_{n-1}^3$

\item[б)]\textbf{(2 балла)} Сколько на нём можно нарисовать фигур $F_2$?  Фигурой $F_2$ следует считать фигуру, состоящую из горизонтальной ``перекладины'' толщины в $1$ клетку, к правому концу которой снизу крепится перекладина высоты не менее $1$ клетки и толщины в $1$ клетку. Сверху к горизонтальной перекладине крепятся три одинаковые вертикальные перекладины высоты не менее $2$ клеток и толщины в одну клетку, причем самая левая из вертикальных перекладин крепится к левому концу горизонтальной, затем между этой (самой левой) вертикальной перекладиной и ближайшей к ней справа имеется как минимум $2$ клетки, потом имеется промежуток как минимум в одну клетку, и наконец самая правая перекладина отстоит от места крепления нижней вертикальной не менее, чем на $2$ клетки. (Приоритет отдавайте рисунку, а не описанию фигуры. Если описание противоречит рисунку - давайте знать, но правильная интерпретация задачи всегда нарисована на рисунке.)
\end{itemize}

\textbf{Решение}

Границы вертикальных линий выберем $C_{n-1}^3$ - $n-1$ потому что расстояние для верхней части минимум $2$, $3$ - $3$ точки (центральная горизонтальная и верх/низ). Границы горизонтальной перекладины - $C_{n-2-1-2}^4$, потому что $2$, $1$, $2$ - расстояния между соседними вертикальными, $4$ - всего $4$ вертикальных линии.
Итого: $C_{n-5}^4{\cdot}C_{n-1}^3$

\textbf{Внимание!} В данной задаче важнее обоснование ответа, а не сам ответ. Очень внимательно прочтите пункт $2$ правил оформления. Нарушение этого пункта чревато серьёзными штрафами при проверке! Полный балл за каждый пункт будет ставиться только в случае получения ответа, не содержащего суммирования по параметру.

\[
\begin{tikzpicture}
\usetikzlibrary{patterns}
\draw[step=0.6,red,very thin] (0,0) grid (4.2,4.2);
\filldraw[pattern=north east lines] (3.6,3.0) -- (1.2,3.0) -- (1.2,0.6) -- (1.8,0.6) -- (1.8,2.4) -- (3.0,2.4) -- (3.0,0.6) -- (3.6,0.6) -- (3.6,3.0);
\filldraw[pattern=north east lines] (3.0,3.0) -- (3.0,4.2) -- (3.6,4.2) -- (3.6,3.0);
\filldraw[pattern=north east lines] (1.2,3.0) -- (1.2,4.2) -- (1.8,4.2) -- (1.8,3.0);
\draw[blue] (1.2,3.0) -- (1.2,3.3);
\draw[blue] (3.6,3.0) -- (3.6,3.3);
\draw[blue] (0.9,3.0) -- (1.2,3.0);
\draw[blue] (0.9,2.4) -- (1.2,2.4);
\draw[blue] (0.9,0.6) -- (1.2,0.6);
\draw[blue] (1.2,0.6) -- (1.2,0.3);
\draw[blue] (1.8,0.6) -- (1.8,0.3);
\draw[blue] (3.0,0.6) -- (3.0,0.3);
\draw[blue] (3.6,0.6) -- (3.6,0.3);
\draw[<->,blue] (1.0,3.0) -- (1.0, 4.2);
\draw[<->,blue] (1.2,0.4) -- (1.8,0.4);
\draw[<->,blue] (1.8,3.2) -- (3.0,3.2);
\draw[<->,blue] (1.0,3.0) -- (1.0,2.4);
\draw[<->,blue] (1.0,2.4) -- (1.0,0.6);
\draw[<->,blue] (3.0,0.4) -- (3.6,0.4);
\draw (2.4,3.3) node {\tiny{$\geqslant$2 кл.}};
\draw (0.9,2.7) node[rotate=90] {\tiny{1кл.}};
\draw (0.9,1.5) node[rotate=90] {\tiny{$\geqslant$2 клетки}};
\draw (1.5,0.3) node {\tiny{1кл.}};
\draw (3.3,0.3) node {\tiny{1кл.}};
\draw (0.9,3.6) node[rotate=90] {\tiny{$\geqslant$ 1 кл.}};
\draw (2.2,-0.3) node {Рис. 1. Пример фигуры $F_1$.};



\filldraw[pattern color=green, pattern=north west lines] (1.2,2.4) -- (1.2,3.0) -- (1.8,3.0) -- (1.8,2.4) -- (1.2,2.4);
\filldraw[pattern color=green, pattern=north west lines] (3.0,2.4) -- (3.0,3.0) -- (3.6,3.0) -- (3.6,2.4) -- (3.0,2.4);

\draw[step=0.6,red,very thin] (5.4,0) grid (10.8,4.2);
\filldraw[pattern=north east lines] (6.0,1.2) -- (10.2,1.2) -- (10.2,0.0) -- (10.8,0.0) -- (10.8,1.8) -- (9.6,1.8) -- (9.6,3.6) -- (9.0,3.6) -- (9.0,1.8) -- (8.4,1.8) -- (8.4,3.6) -- (7.8,3.6) -- (7.8,1.8) -- (6.6,1.8) -- (6.6,3.6) -- (6.0,3.6) -- (6.0,1.2);
\draw[blue] (6.0,1.2) -- (5.7,1.2);
\draw[blue] (6.0,1.8) -- (5.7,1.8);
\draw[blue] (6.0,3.6) -- (5.7,3.6);
\draw[blue] (10.2,0.0) -- (5.7,0.0);
\draw[<->,blue] (5.8,1.2) -- (5.8,1.8);
\draw[<->,blue] (5.8,1.8) -- (5.8,3.6);
\draw[<->,blue] (5.8,1.2) -- (5.8,0.0);
\draw (5.7,0.6) node[rotate=90] {\tiny{$\geqslant$1 кл.}};
\draw (5.7,1.5) node[rotate=90] {\tiny{1кл.}};
\draw (5.7,2.7) node[rotate=90] {\tiny{$\geqslant$2 кл.}};
\draw[blue] (6.0,3.6) -- (6.0,3.9);
\draw[blue] (6.6,3.6) -- (6.6,3.9);
\draw[blue] (7.8,3.6) -- (7.8,3.9);
\draw[blue] (8.4,3.6) -- (8.4,3.9);
\draw[blue] (9.0,3.6) -- (9.0,3.9);
\draw[blue] (9.6,3.6) -- (9.6,3.9);
\draw[blue] (10.2,1.8) -- (10.2,3.9);
\draw[blue] (10.8,1.8) -- (10.8,3.9);
\draw[<->,blue] (6.0,3.8) -- (6.6,3.8);
\draw[<->,blue] (6.6,3.8) -- (7.8,3.8);
\draw[<->,blue] (7.8,3.8) -- (8.4,3.8);
\draw[<->,blue] (8.4,3.8) -- (9.0,3.8);
\draw[<->,blue] (9.0,3.8) -- (9.6,3.8);
\draw[<->,blue] (9.6,3.8) -- (10.2,3.8);
\draw[<->,blue] (10.2,3.8) -- (10.8,3.8);
\draw (6.3,3.9) node {\tiny{1кл.}};
\draw (7.2,3.9) node {\tiny{$\geqslant$2кл.}};
\draw (8.1,3.9) node {\tiny{1кл.}};
\draw (8.7,3.9) node {\tiny{$\geqslant$1кл.}};
\draw (9.3,3.9) node {\tiny{1кл.}};
\draw (9.9,3.9) node {\tiny{$\geqslant$2кл.}};
\draw (10.5,3.9) node {\tiny{1кл.}};
\filldraw[pattern color=green, pattern=north west lines] (7.8,1.2) -- (7.8,1.8) -- (8.4,1.8) -- (8.4,1.2) -- (7.8,1.2);
\filldraw[pattern color=green, pattern=north west lines] (9.0,1.2) -- (9.0,1.8) -- (9.6,1.8) -- (9.6,1.2) -- (9.0,1.2);
\filldraw[pattern color=violet, pattern=north west lines] (6.0,1.2) -- (6.0,1.8) -- (6.6,1.8) -- (6.6,1.2) -- (6.0,1.2);
\filldraw[pattern color=yellow, pattern=north west lines] (10.2,1.2) -- (10.2,1.8) -- (10.8,1.8) -- (10.8,1.2) -- (10.2,1.2);
\draw (8.1,-0.3) node {Рис. 2. Пример фигуры $F_2$.};
\end{tikzpicture}\]}

\task{2}{В эксперименте по бросанию монетки было проведено $m + n$ испытаний (бросков монетки), при этом $m$ раз выпал орел, а $n$ - решка. Известно, что если при $i$-том броске выпала решка (или орел), то как минимум в одном из бросков $i-1$ или $i+1$ также выпала решка (соответственно, орел). Результатом эксперимента будем считать упорядоченный набор исходов. Найдите число всех возможных результатов. Ответ можете оставить в виде суммы.}

\textbf{Решение}

Представим исходы подброса монетки в виде $0$ и $1$ в зависимости орла или решки, пусть $0$ - орел, $1$ - решка. Все $n+m$ исходов запишутся в строку из $0$ и $1$, различные исходы всех бросков будут соответствовать различным строкам. Всего нулей $m$, единиц - $n$. Теперь условие ограничения перепишется как ``в результирующей строке не должна стоять отдельно единица или ноль''. 
В итоговой строке схлопнем подряд идущие нули и единицы в соответствующую цифру ($000$ $\rightarrow$ $0$) и будем считать лишь количество единиц и нулей в строке. Понятно, что разница между ними по модулю будет $\leqslant1$.
\begin{enumerate}
\item[a)]
    Количество нулей и единиц совпадает. В этом случае будем строить строку с $r$ группами единиц (и нулей) следующим образом: поставим в каждую группу по $2$ единицы, затем будем доставлять оставшиеся единицы по $r$ группам - это сочетание с повторениями. Точно так же разложим нули. Поскольку оценка сверху для $r$ равна $M=min(\lfloor \frac{n}{2} \rfloor, \lfloor \frac{m}{2} \rfloor)$, потому что иначе невозможно поставить по $2$ элемента в группе изначально, то в общем это даст (множитель $2$, потому что можно сделать $010101$ или $101010$, например, то есть можно поменять элементы на противоположные и ничего не поменяется):
    $2 \cdot \sum_{r=1}^{M} C_{r+(n-2 \cdot r)-1}^{r-1} \cdot C_{(r-1)+(m- 2 \cdot r)-1}^{r-1}$
    
\item[b)]
    Если разница между количеством единиц и нулей равна $1$, то ограничение сверху на $r$ будет $M1=min(\lfloor \frac{n}{2} \rfloor+1, \lfloor \frac{m}{2} \rfloor)$, если единиц больше, или $M0=min(\lfloor \frac{n}{2} \rfloor, \lfloor \frac{m}{2} \rfloor+1)$, если больше нулей. Тогда (в случае, когда единиц больше, для нулей будет аналогично с точностью до симметрии) количество способов для расстановки равно (здесь $r$ - количество единиц):
    $\sum_{r=1}^{M} C_{r+(n-2 \cdot r)-1}^{r-1} \cdot C_{(r-1)+(m-2 \cdot (r-1))-1}^{r-2}$
    
Итого в сумме будет: $\sum_{r=1}^{M1} C_{r+(n-2 \cdot r)-1}^{r-1} \cdot C_{(r-1)+(m-2 \cdot (r-1))-1}^{r-2}+2 \cdot \sum_{r=1}^{M} C_{r+(n-2 \cdot r)-1}^{r-1} \cdot C_{(r-1)+(m- 2 \cdot r)-1}^{r-1}+\sum_{r=1}^{M0} C_{(r-1)+(n-2 \cdot (r-1))-1}^{r-2} \cdot C_{r+(m-2 \cdot r)-1}^{r-1}$
\end{enumerate}

\task{}{\begin{itemize}
\item[а)]\textbf{(2 балла)}	Пусть элементы множества $A = \{1, 2, \dots, n\}$ записаны в порядке возрастания по окружности. Для $0 \le k \le [\frac{n}{2}]$ найдите число $k$ элементных подмножеств множества $A$, у которых никакие два элемента не являются соседями на окружности.

\textbf{Решение}

Для начала попробуем найти количество способов выбрать $k$ элементов из $n$-элементного \emph{ряда} так, чтобы никакие два не стояли рядом. Для этого выберем $n-k$ чисел, и попытаемся ставить $k$ элементов в промежутки, учитывая пространство перед первым и после последнего. Всего таких промежутков $n-k+1$, поэтому количество искомых расстановок равно $C_{n-k+1}^k$.\\
Теперь рассмотрим некоторый элемент на окружности - пусть, не нарушая общности, это будет $1$. Тогда все способы выбора подмножества размера $k$ по условию разобьются на $2$ случая: когда элемент $1$ выбран и когда нет. В случае, когда он выбран, соседние элементы ($n$ и $2$) точно выбраны не будут, поэтому оставшиеся элементы можно развернуть в ряд размера $n-3$ с необходимостью выбрать еще $k-1$ элемент - это можно сделать $C_{(n-3) - (k-1) + 1}^{k-1} = C_{n-k-1}^{k-1}$. Во втором случае, когда элемент $1$ не выбран, опять можно развернуть оставшиеся элементы в ряд размером $n-1$ с необходимостью выбрать еще $k$ элементов, что можно сделать $C_{(n-1)-k+1}^k = C_{n-k}^k$.\\ 
Итого, можно количество способов выбрать элементы по условию равно $C_{n-k-1}^{k-1} + C_{n-k}^k$.

\item[b)]\textbf{(2 балла)}	Пусть есть $n$ пар муж-жена, и круглый стол с $2n$ стульями. Положим, что $n$ жен уже сели, причем так, что между любыми двуями соседними женами есть ровно один свободный стул - т.е. имеет место чередование пустых стульев и занятых женами. Покажите, что число рассадок $n$ мужей на оставшихся стульях, где \emph{\textbf{ровно}} $r$ мужей, $0 \leqslant r \leqslant n$, непосредственно соседствуют со своими женами, равно $\sum\limits_{k = r}^n(-1)^{k-r}C_k^r \frac{2n}{2n -k} C_{2n - k}^k (n - k)!$
\end{itemize}

\textbf{Внимание!} При оформлении задачи учтите пункт $5$ правил, написанных на странице курса на ШАД-вики. Если он проигнорирован, полный балл Вы не получите!}


\task{2}{Покажите, что в произвольном наборе из пяти натуральных чисел найдутся $3$ числа, сумма которых делится на $3$.}

\textbf{Решение}

Заметим, что если какие-то $3$ числа имеют одинаковый остаток деления на $3$, то в сумме они дадут число, делящееся на $3$. Действительно, пусть остаток каждого числа равен $r$, тогда в сумме будет остаток $3r$, что делится на $3$. Тогда, если никакие $3$ числа не имеют одинаковый остаток, каждого остатка не более $2$, но в этом случае из $5$ чисел найдется $3$ числа с разными остатками, потому что иначе (принцип Дирихле) надо рассадить $5$ кроликов (числа) в $2$ клетки (в клетке числа с разными остатками). Но тогда есть $3$ числа с остатками {$-1$, $0$, $1$}, и в сумме они дадут число, делящееся на $3$.

\task{2}{В групповом этапе крупного турнира в группе $A$ играет $10$ команд. Формат этапа - round robin, т.е. каждая команда проводит в точности одну встречу с каждой командой группы. Согласно правилам, за победу команда получает $1$ очко, за ничью не получает ничего, а за поражение теряет $1$ очко. Когда групповой этап закончился, оказалось что более, чем $70$\% встреч окончилось ничьей. Покажите, что в группе $A$ есть две команды, набравшие одинаковое число очков.}

\textbf{Решение}

Посчитаем, что как максимум $13$ ($45\cdot0.3)$ игр в турнире могли закончиться не вничью, соответственно, во всех остальных играх очков не было набрано. Теперь пойдем от противного: пусть по итогам турнира нет двух команд с равным количеством очков. Рассмотрим два случая:

\begin{itemize}
\item[$1$)] У одной команды $0$ очков. Тогда оставшиеся команды разобьем на $2$ класса: те, которые набрали положительное количество очков (класс+) и те, которые набрали отрицательное количество очков (класс-). Поскольку после любого исхода игры общая сумма набранных очков равна $0$, то сумма итоговых очков также должна быть равна $0$. Это значит, что сумма очков класса+ должна быть по модулю равной сумме очков класса-. Понятно, что модуль суммы очков у класса+ и у класса- должен не превосходить $13$ у каждого - иначе получится, что одним из классов выиграно больше $13$ количество матчей, что неверно. Отсюда следует, что в каждом классе может быть не более $4$ команд, если хотя бы $5$, то с условием различия очков у каждой команды будет минимум $1+2+3+4+5=15$ очков в сумме у класса, что больше $13$. С другой стороны, если в каждом классе не более $4$х команд, то в сумме команд не более $4+4+1=9<10$ - противоречие. Значит такого случая быть не может.


\item[$2$)] Нет команды с $0$ очков. Тогда, как и в предыдущем пункте, есть разбиение на класс+ и класс-, модули суммы в них совпадают, но и количество команд в каждом классе не более $4$ по аналогичным рассуждениям, что дает не более, чем $8$ команд - опять противоречие.
\end{itemize}}

Таким образом, невозможно получить турнир, в котором при максимальных $13$ результативных играх не найдется пара команд с общим количеством набранных очков.

\task{3}{Подстрокой строки $s = c_1c_2\dots c_n$ называется последовательность символов $t = c_i c_{i+1} \dots c_{i + k}, \; 1 \le i \le i + k \le n$. Рассмотрим строки длины 15, которые состоят только из 0 и 1. Найдите число таких строк, что в них есть ровно две подстроки $00$, пять подстрок $11$, три подстроки $01$, четыре подстроки $10$. (Еще раз обратите внимание на п.5 правил, написанных на странице курса на ШАД-вики.)}

\textbf{Решение}

Сначала расставим $10$ и $01$ по указанным их количествам. Это можно сделать единственным образом $10101010$. Теперь будем доставлять два $0$: понятно, что их можно ставить везде, кроме как в начало строки - тогда появится еще одна $01$, при этом поставив справа или слева от $1$ дает ту же строку, поэтому фактически есть $4$ места справа от $1$ для подстановки. Это дает от сочетания с повторениями $C_{4+2-1}^{2}=C_5^2$. Теперь с таким же правилом попробуем расставить $1$: опять же, расставлять надо не в самый последний элемент, иначе появится еще $01$, а выборов остается $4$. В итоге $C_{4+5-1}^{5}=C_8^5=C_8^3$. Общий результат: $C_5^2+C_8^3$.

\task{2}{Найдите асимптотику для коэффициента $C_{2n}^{\left[\frac{n}{3}\right]}$ при $n\to +\infty$. Здесь $[x]$ --- целая часть действительного числа $x$.

\textbf{Внимание!} В задаче требуется найти функцию $f(n)$, такую, что $\lim_{n\to+\infty}\frac{f(n)}{C_{2n}^{\left[\frac{n}{3}\right]}}=1$, а не запись биномиального коэффициента в виде $(\alpha+o(1))^n$. Можно рассмотреть отдельно случаи, когда остатки от деления на $3$ числа $n$ равны $0,1$ и $2$ и вычислить ответ в каждом из случаев. Ответ в каждом из этих случаев должен быть записан в виде формулы, не содержащей знаков целой части числа или введённых в процессе решения вспомогательных функций и параметров.}

\textbf{Решение}

$C_{2n}^{\left[\frac{n}{3}\right]}=\displaystyle\frac{(2n)!}{(2n-\left[\frac{n}{3}\right])!(\left[\frac{n}{3}\right])!}=\displaystyle\frac{(2n)!}{(\frac{5n}{3}+\epsilon)!(\frac{n}{3}-\epsilon)!}\sim\displaystyle\frac{(2n)^{2n}\sqrt{4\pi{n}}}{\sqrt{2\pi(\frac{5n}{3}+\epsilon)}\sqrt{2\pi(\frac{n}{3}-\epsilon)}(\frac{5n}{3}+\epsilon)^{\frac{5n}{3}+\epsilon}(\frac{n}{3}-\epsilon)^{\frac{n}{3}-\epsilon}}\sim\\
\sim\frac{(2n)^{2n}\sqrt{n}}{2\sqrt{\pi}\sqrt{\frac{5n}{3}+\epsilon}\sqrt{\frac{n}{3}-\epsilon}(\frac{5n}{3})^{\frac{5n}{3}}(\frac{5n}{3}e)^{\epsilon}(\frac{n}{3})^{\frac{n}{3}}(\frac{n}{3}e)^{-\epsilon}}=\frac{(2n)^{2n}\sqrt{n}}{2 \cdot 5^{\epsilon}\sqrt{\pi}\sqrt{\frac{5n}{3}+\epsilon}\sqrt{\frac{n}{3}-\epsilon}(\frac{5n}{3})^{\frac{5n}{3}}(\frac{n}{3})^{\frac{n}{3}}}=\\
\frac{6^{2n}\sqrt{n}}{2 \cdot 5^{\frac{5n}{3}+\epsilon}\sqrt{\pi}\sqrt{\frac{5n}{3}+\epsilon}\sqrt{\frac{n}{3}-\epsilon}}$\\

Что и является искомой асимптотикой

\task{2}{Дано произвольное натуральное число $k\ge 1$, то есть предполагается, что $k$ --- произвольная заданная константа. Вычислите асимптотику полиномиального коэффициента $P(5n,3n+k,n+k,k)=$ $\frac{(9n+3k)!}{(5n)!(3n+k)!(n+k)!k!}$ при $n\to+\infty$.

\textbf{Внимание!} В задаче требуется найти функцию $f_k(n)$, такую, что $\lim_{n\to+\infty}\frac{f_k(n)}{P(5n,3n+k,n+k,k)}=1$, а не запись биномиального коэффициента в виде $(\alpha+o(1))^n$. Полный балл будет ставиться только в случае, если ответ досчитан до конца, то есть до формулы, которую нельзя упростить. Ответ должен быть записан в виде формулы, не содержащей введённых в процессе решения вспомогательных функций и параметров. За расписывание ассимптотик констант по формуле Стирлинга будет ставиться $0$ баллов - это очень грубая ошибка!}

\textbf{Решение}

$\displaystyle{P(5n,3n+k,n+k,k)=\frac{(9n+3k)!}{(5n)!(3n+k)!(n+k)!k!}\sim\frac{9^{9n+3k}n^{9n+3k}(1+\frac{k}{3n})^{9n+3k}}{5^{5n}n^{5n}3^{3n+k}n^{3n+k}n^{n+k}(1+\frac{k}{3n})^{3n+k}(1+\frac{k}{n})^{n+k}k^k}}*\\
*\sqrt{\frac{2\pi(9n+3k)}{2\pi5n\cdot2\pi(3n+k)\cdot2\pi(n+k)\cdot2\pi{k}}}=\frac{3^{15n+5k}n^{k}e^{\frac{k(3n+k)}{n}}}{5^{5n}e^{\frac{k(3n+k)}{3n}}e^{\frac{k(n+k)}{n}}k^k}\cdot\frac{1}{2\pi}\sqrt{\frac{9n+3k}{2\pi5n(3n+k)(n+k)k}}=\frac{3^{15n+5k}n^{k}e^{\frac{k(3n-k)}{3n}}}{5^{5n}k^k}*\\
*\frac{1}{2\pi}\sqrt{\frac{9n+3k}{\pi10n(3n+k)(n+k)k}}$

\task{2}{Найдите асимптотику величины $C_{n^8+2n^4}^{n^4}$ при $n\to\infty$.

\textbf{Решение}

$\displaystyle C_{n^8+2n^4}^{n^4}=[t=n^4]=C_{t^2+2t}^{t}=\frac{(t^2+2t)!}{t!(t^2+t)!}\sim\frac{\sqrt{2\pi(t^2+2t)}(\frac{t^2+2t}{e})^{t^2+2t}}{\sqrt{2\pi}\sqrt{t}\sqrt{2\pi}\sqrt{t^2+t}(\frac{t^2+t}{e})^{t^2+t}(\frac{t}{e})^t}=\frac{\sqrt{t+2}(\frac{t+2}{e})^{t^2+2t}}{\sqrt{2\pi}\sqrt{t}\sqrt{t+1}(\frac{t+1}{e})^{t^2+t}(\frac{1}{e})^t}=\frac{1}{\sqrt{2\pi}}\sqrt{\frac{t+2}{t^2+t}}(\frac{t+2}{t+1})^{t^2+t}(t+2)^t=\frac{1}{\sqrt{2\pi}}\sqrt{\frac{t+2}{t^2+t}}(1+\frac{1}{t+1})^{(t+1)t}(t+2)^t=\frac{1}{\sqrt{2\pi}}\sqrt{\frac{t+2}{t^2+t}}e^t(t+2)^t=\\
=\frac{1}{\sqrt{2\pi}}\sqrt{\frac{n^4+2}{n^8+n^4}}e^{n^4}(n^4+2)^{n^4}


\textbf{Внимание!} В задаче требуется найти функцию $f(n)$, такую, что $\lim_{n\to+\infty}\frac{f(n)}{C_{n^8+2n^4}^{n^4}}=1$, а не запись биномиального коэффициента в виде $(\alpha+o(1))^n$. Полный балл будет ставиться только в случае, если ответ досчитан до конца, то есть до формулы, которую нельзя упростить. Ваш ответ не должен содержать неопределенностей вида $(n^8+o(1))^{n^8}$ и подобных им - за это Вы получите 0 баллов. Ответ должен быть записан в виде формулы, не содержащей введённых в процессе решения вспомогательных функций и параметров.}

\task{2}{Положим $s(n)=\min\{m\in\NN, m\neq 1\mid C_n^m\cdot e^{-\frac{m^5}{(\ln m)^{2}\ln\ln m}}<1\}$. Найдите асимптотику функции~$s(n)$. Ответ необходимо максимально упростить.}

\textbf{Решение}

$m=\sqrt{n}$:

$C_n^m\sim\frac{n^m}{m!} \sim e^{\frac{m^5}{(\ln(m))^2 \ln\ln(m)}}$;

$m\ln(n) - \frac{1}{2}\ln(2)\pi m - m\ln(m) + m \sim \frac{m^5}{(\ln(m))^2 \ln\ln(m)}$;

$\ln(n) \sim \frac{m^4}{(\ln(m))^2 \ln\ln(m)} \implies m^4 \sim \ln(n) \cdot (\ln(m))^2 \cdot \ln(\ln(m))$(1)

$\ln\ln(n) \sim 4\ln(m) - 2\ln\ln(m) - \ln\ln\ln(m) \sim 4\ln(m)$ (2)

$\ln\ln\ln(n) \sim \ln 4 + \ln\ln(m) \sim \ln\ln(m)$ (3)

Подставим (2) и (3) в (1)

$m^4 \sim \ln(n) \cdot (\frac{\ln\ln(n)}{4})^2 \cdot \ln\ln\ln(n)$

$m \sim \sqrt[4]{\ln n \cdot (\frac{\ln\ln(n)}{4})^2 \cdot \ln\ln\ln(n)}$

\task{2}{Найдите асимптотику для функции $x(n)=\max\left\{x\in\mathbb{N}:x^{x^2\cdot x!}\leqslant n\right\}.$ Ответ необходимо максимально упростить.}

\textbf{Решение}

${\displaystyle}n=x(n)^{\displaystyle{x^2(n)\cdot{x}(n)!}}+r(s)$\\

$\displaystyle0{\leqslant}r(s){\leqslant}(x(n)+1)^{\displaystyle(x(n)+1)^2\cdot(x(n)+1)!}$\\

$\displaystyle\ln(n)=x^2(n){\cdot}x(n)!{\cdot}\ln(x(n))+\ln(1+\frac{r(s)}{x(n)^{\displaystyle{x^2(n){\cdot}x(n)!}}})$;\\

$\displaystyle{x^2(n){\cdot}x(n)!{\cdot}\ln(x(n))}{\sim}x^3(n){\cdot}x(n)!\cdot\ln(x(n))-x(n)+\ln(\sqrt{2{\pi}x(n)}){\sim}x^3(n){\cdot}x(n)!\ln(x(n))(1+o(1))$;\\

$\displaystyle0\leqslant\ln(1+\frac{r(s)}{x(n)^{x^2(n){\cdot}x(n)!}}){\leqslant}\ln(1+\frac{(x(n)+1)^{\displaystyle(x(n)+1)^2\cdot(x(n)+1)!}-x(n)^{\displaystyle{x^2(n){\cdot}x(n)!}}}{x(n)^{\displaystyle{x^2(n){\cdot}x(n)!}}})=$\\

$=\displaystyle\ln(\frac{(x(n)+1)^{\displaystyle(x(n)+1)^2\cdot(x(n)+1)!}}{x(n)^{\displaystyle{x^2(n){\cdot}x(n)!}}})=(x(n)+1)^2\cdot(x(n)+1)!\cdot\ln(x(n)+1)-x^2(n){\cdot}x(n)!\cdot\ln(x(n))=\\$

$=x(n)!((x(n)+1)^2\cdot(x(n)+1)\cdot\ln(x(n)+1)-x^2(n){\cdot}\ln(x(n)))=(x(n)+1)^2\cdot(x(n)+1)!\cdot\ln(x(n)+1)(1+o(1))$, потому что в первом слагаемом есть $x(n)+1$\\

Тогда:\\

$\ln(n){\sim}x^3(n){\cdot}x(n)!\cdot\ln(x(n))(1+o(1))+(x(n)+1)^2\cdot(x(n)+1)!\cdot\ln(x(n)+1)(1+o(1))=x(n)!(1+o(1))(x^3(n)\cdot\ln(x(n))+(x(n)+1)^3\cdot\ln(x(n)+1)){\sim}(x(n)+1)^3{\cdot}x(n)!{\cdot}\ln(x(n))$\\

Логарифмируем:

$\ln(\ln(n)) \sim 3\ln(x(n)+1)+\ln(x(n)!)\sim3\ln(x(n)+1)+x(n)\cdot\ln(x(n))-x(n)\sim x(n)\cdot\ln(x(n))$;\\

$\ln(\ln(\ln(n)))=\ln(x(n))+\ln(\ln(x(n)))\sim\ln(x(n))\implies x(n) \sim \ln(\ln(n))$ - ответ

\end{document} 